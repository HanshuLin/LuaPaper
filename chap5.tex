\newcommand{\desugar}[2]{{#1} & \xlongequal[]{def} & {\begin{array}{@{}l@{}} #2 \end{array}}\\}
\newcommand{\desugarline}[1]{ & & {\begin{array}{@{}l@{}} #1 \end{array}}\\}

\newcommand{\translate}[2]{\llbracket {#1} \rrbracket & \xlongequal[]{} & {\begin{array}{@{}l@{}} #2 \end{array}}\\}
\newcommand{\translateline}[1]{ &  & {\begin{array}{@{}l@{}} #1 \end{array}}\\}

\chapter{Lua to Feather Weight Lua}
\label{chp:luaTranslation}

We have introduced the full syntax and semantics of FWLua in Chapter~\ref{chp:CoreFWLua}. We can see that FWLua is a much smaller language than the full version of Lua, with many differences between them.
However, our purpose is to let FWLua capture the essence of Lua with only these core features.
In this chapter, we discuss how FWLua works like Lua by showing how syntactic desugaring can be used to convert between FWLua and Lua. Since there are many additional features in Lua, we first discuss how to desugar in Lua to let the code fit FWLua. Secondly we give some translation rules from Lua to FWLua.

There are several important parts we focus on in desugaring Lua. First of all, getting variables is taken into consideration. Secondly, Lua allows many different ways of representing variable updates and declarations in different conditions and scopes. We also consider functions because there are some different qualities for functions in the full Lua language. Tables, in addition, need to be discussed since they are the primary data structure with metamethods and metatables. Finally, we discuss how we may desugar control statements. 

We will discuss those parts in detail, along with examples, in the following sections.


\section{Variable select}
Selecting variables is used for computations to get a user defined value. Generally speaking, getting variables in a common programming language is easy and straightforward: users simply write a variable as the key for getting its value. For example, the following expressions used in Lua would stand for ``getting a value'':

\begin{verbatim}
x ---get the value of x
t["x"] ---value of key "x" in table t
t.y ---value of key "y" in table t
\end{verbatim}

Every variable in Lua has its own field. 
The first line of the example has no associated table.
To normalize it, we need to give a table name to that variable to make it clear.
Lua provides a reserved table, {\tt \_ENV} to represent the global table. Therefore, every variable now can have a table.

The normal format, in Lua, is {\tt e["e"]}. 
The expression {\tt t.y} is syntactic sugar with key {\tt y} since the key is the string type in the table. After we desugar these expressions, all variable select expressions in Lua can be represents as the format {\tt e[e]}, such as the following:

\begin{verbatim}
_ENV["x"] ---get the value of x
t["x"] ---value of key "x" in table t
t["y"] ---value of key "y" in table t
\end{verbatim}

Lua provides metamethods and metatables in table constructors. The metamethod {\tt \_\_index} points to a fallback table. If the key that users want to index is not available in the current table, this metamethod is triggered. However, there is a reserved function {\tt rawget(t, k)} in Lua, which will index the key {\tt k} in the table {\tt t} and ignore the hook in its metatable.

\section{Variable update \& declaration}\label{sec:desugarVarUpdate}
Variable update and declaration are two key procedures in a programming language. 
In these two different procedures, one or more variables are declared or updated in the specific scope. There are many ways of defining variables in Lua, even including declaring multiple variables in one statement. Before we discuss desugaring, the following example gives each way of doing variable declaration and updating.

%FIXED: Not clear what the difference is between the first two cases
\begin{verbatim}
x = 11 ---declare a variable x
x = 12 ---update x since it has existed
t.y = 63 ---key variable declarations in table
a,b,c,d = 1,2+3,4,5 ---multiple variable declarations
local f = "Hello" ---variable declaration in scope
\end{verbatim}

Direct table select in Lua is treated as a syntactic sugar, and we desugar it to computed table select. 
Multiple variable declarations can be represented using a set of single assignment statements. 
The local variable declaration is more complicated. Lua actually finds out which scope, or which table, should the variable be set in a fixed order.
To handle local variables,
we make a reserved table for every defined scope to store local variables in FWLua. Furthermore, FWLua will transfer every scope in Lua into a table, including blocks, functions and global variables. We define a reserved table in FWLua named {\it \_local} to provide local scope, and a reserved table {\it \_outer} to provide outer scope in FWLua.

There is also a reserved function in Lua to get rid of effects of metatables in setting variables, which is {\tt rawset(t, k, v)}. This is the reason we build our syntax in FWLua using this format, and will then desugar Lua to FWLua, followed by telling mechanisms in Lua.
Figure~\ref{fig:desLuaVar} shows how we desugar syntax about variables in Lua, using FWLua.
An example of decoding assignments in Lua to FWLua is shown below.
\begin{verbatim}
local x = 12  -- Lua, in the scope of a
rawset(rawget(a, "_local"), "x", 12) --desugared in FWLua
\end{verbatim}

\begin{figure}
\caption{Desugaring Lua Variables}\label{fig:desLuaVar}
{\bf Variable Select:}\[
\begin{array}{rclcl}
\desugar{e_1[e_2]}{{\tt if}~ e_1[e_2] ~\sim= {\tt nil} ~ {\tt then}\\}
\desugarline{ 
     \quad{\tt rawget}(e_1, e_2) ~\\
     {\tt else}\\
     \quad{\tt local}~f = {\tt getmetatable}(e_1).{\tt{\_}{\_}index};\\
     \quad f(e_1,e_2)\\ 
     {\tt end}
     }
\desugar{e_1.e_2}{e_1[e_2]}
\desugar{e.x}{e[``x"]}
\desugar{x}{a.x}
\end{array}\]

{\bf Variable Declaration \& Update:}\[
\begin{array}{rclcl}
\desugar{e_1[e_2]=e_3}{
    {\tt if}~ e_1[e_2] ~== {\tt nil} ~ {\tt then}\\
}
\desugarline{  
     \quad{\tt rawset}(e_1, e_2,e_3) ~\\
     {\tt else}\\
     \quad{\tt local}~f = {\tt getmetatable}(e_1).{\tt{\_}{\_}newindex};\\
     \quad f(e_1,e_2,e_3)\\ 
     {\tt end}\\
}
\desugar{e_1.x=e_2}{e_1[``x"] = e_2}
\desugar{\overline{e_i}=\overline{e_j}}{\overline{e_i = e_j;}} %FIXME: Need to discuss
\desugar{{\tt local}~x=e~(in~a)}{a.\_local.x = e;} %FIXME: Need to discuss

\end{array}\]
\end{figure}

\section{Tables and Metatables}\label{sec:TranslateTabls}
Based on the full version of Lua, table constructors are used as expressions to assign a set of pairs with key and value in the specific table. Instead of it, there is only one syntax in FWLua called ``new allocation'' to allocate and then be evaluated as a new address in the store. 
However, the syntax {\tt rawset(t,s,v)} in FWLua will also be evaluated as an address, just as this function does in Lua.  We thus can assign pairs with values and keys, or changing behaviors using some kinds of expressions about this table. In other words, we will completely decompose the whole syntax of ``table constructor'' in to a set of single expressions in FWLua. 

To make it more concrete, the following code is from the full version of Lua using table constructor:

\begin{verbatim}
table = {x=1, y=2}
\end{verbatim}

Basically, Lua constructs a table without any metatable by default. In FWLua, we represent this example by a set of rawset statements, just like the following:

\begin{verbatim}
rawset(_ENV, "table", {}) ---new allocation
rawset(rawget(_ENV, "table"), "x", 1)
rawset(rawget(_ENV, "table"), "y", 2)
\end{verbatim}

Or, because rawset in FWLua can also return that address, we can use a sequence of rawset expression, like a recursive funtion calls:

\begin{verbatim}
rawset(rawset(rawset(_ENV, "table", {}) , "x", 1), "y", 2)
\end{verbatim}

However, metatables and metamethods have to be considered. In Section~\ref{sec: LuaTable}, we have discussed that tables can be used as different data structures by changing metatables or defining some functions. Therefore, we reserve a table with the name of {\tt \_metatable} to represent metatables in Lua. There are also two functions, {\tt setmetatable(t, mt)} and {\tt getmetatable(t)}, highly related to table constructor. We can desugar them by using {\tt rawset}, {\tt rawget}, and {\tt \_metatable} in FWLua.

%FIXME: Interesting point, but we need to think on the wording.  Let's discuss.
%FIXME: Is this really true?  I thought that you could chain rawsets?
Figure~\ref{fig:desLuaTable} show how to translate Lua into FWLua about table constructors. It is noteworthy that, what a rawset expression in FWLua returns is an address, while an assignment statement in Lua returns nothing. In other words, we simplified the algorithm to make it better for the research in Lua.

\begin{figure}
\caption{Desugaring Lua Table}\label{fig:desLuaTable}
{\bf Table:}\[
\begin{array}{rclcl}
\desugar{t = \{\}}{{\tt rawset}(a,``t", \{\})}
\desugar{t = \{\overline{e_i = v_j}\}}{{\tt rawset}(...{\tt rawset}({\tt rawset}(a,``t", \{\}), e_1, v_1)...)}
\desugar{{\tt setmetatable}(t, mt)}{{\tt rawset}(a,``\_metatable", mt)}
\desugar{{\tt getmetatable}(t)}{{\tt rawget}(a,``\_metatable")}
\desugar{t_1~{\it op}~t_2}{(({\tt rawget}({\tt rawget}(a,``\_metatable"),~{\it op}))(t_1))(t_2)}
\end{array}\]
\end{figure}

\section{Functions and control structures}\label{sec:TraFunc}
Before desugaring functions in Lua, we first introduce those differences between functions in Lua and in FWLua. Since functions in FWLua are similar to lambda calculus, there is an obvious difference, which is that each function in FWLua always take exactly one argument as its perimeter. Therefore, we are going to make every function in Lua as the style of multiple functions in FWLua, or in lambda calculus. For instance, consider a simple Lua function:

\begin{verbatim}
function (x,y) return e end
\end{verbatim}

An equivalent function can be written in a style closer to FWLua as follows:

\begin{verbatim}
function (x) return 
  function (y) return 
    e
  end
end
\end{verbatim}

%FIXME: Note that applying these functions is slightly different, so show that as well.

Return statements in Lua functions are optional, which means a function might not return a value. However, return statements are required in every function in FWLua.
Furthermore, a function in Lua might have its own scope and supports local variables. 

Since we don't provide local variables to functions in FWLua, we now treat the local scope as a special table.
To capture additional information about Lua functions,
we translate every function in Lua to a table with fields for basic information about the function. There are four reserved variables that must be taken into the consideration when we translate a function in Lua into FWLua: {\tt \_local} is a table for variables in the local enviroment; {\tt \_arg} is also a table storing the arguments of the function; {\tt \_outer} is an address pointing to the outer scope; and {\tt \_call} is a function to store the body of the function. There is an example in Appendix~\ref{app:function}.

Lambda function applications can be used to represent a sequence of  statements. We therefore translate sequences of statements into functions in FWLua. Lambda functions can also be used to represent control structures and boolean values, which we discuss in the next section.

Figure~\ref{fig:desLuaFunc} gives the rules for translating functions. As above, the results of the translations are all in FWLua syntax, or in other constructs that we define.

\subsection{Control structures}
Control structures can be defined in terms of a set of lambda functions. 
Following the standard pattern used in the lambda calculus,
we can treat the constant {\tt true} as a function taking 2 parameters\footnote{
  More strictly speaking, {\tt true} is a function that takes one parameter
  and returns another function that takes one parameter and returns
  the argument passed to the outer function.
}
and returning the first value, and treat {\tt false} as a function taking 2 parameters and returning the second value. The syntax is as follows:

\begin{verbatim}
function (x) return 
  function (y) return 
    x
  end
end  --True

function (x) return 
  function (y) return 
    y
  end
end  --False
\end{verbatim}

Using these constructs for true and false, we now can translate the basic control structure in Lua, the {\tt if} statement, into a complicated reserved lambda function in FWLua. The reference~\cite{TAPL} shows us the details about translating control structure statements into lambda functions.

According to our analysis, all forms of loop statements in Lua can be represented by using a set of conditional statements and recursive function calls. For example:

\begin{verbatim}
x = 0;
while x ~= 10 do x = x+1; end
\end{verbatim}

The above code is a loop statement. It increments the variable {\tt x} by one each time, until it equals 10. We can see that, a recursive computation and a conditional statement are in the loop: the program tests whether {\tt x} is equal to 10, and then recursively executes this computation. Therefore, they can be be represent as the following:

\begin{verbatim}
function while()
  if x~=10 do
    x = x+1; while()
  else
    return nil
  end
end;
while()
\end{verbatim}

Hence all different kind of loop statements can be desugared using conditional statements. Figure~\ref{fig:desLuaFunc} gives other desugaring rules.

\begin{figure}
\caption{Desugaring Lua Functions}\label{fig:desLuaFunc}
{\bf Function:}\[
\begin{array}{rclcl}
\desugar{e_1{\tt ;}~e_2}{(\aFunction{a}{e_2})(e_1)}
\desugar{{\tt function}~f(x,y)~e~{\tt end}}{f = \{ {\tt ``\_local"} = \{\}, {\tt ``\_outer"} = a,}
\desugarline{{\tt ``\_arg"} = \{x, y\}, {\tt ``\_call"} = \aFunction{x}{\aFunction{y}{e}}\}}
%FIXME: Not following what ... stands for here.
\desugar{f(\overline{e_i})}{(...(f.\_call)(f.\_arg = e_1)...)(f.\_arg = e_i)}
\end{array}\]

{\bf Control Structure:}\[
\begin{array}{rclcl}

\desugar{\tt True}{\aFunction{x}{\aFunction{y}{x}}}
\desugar{\tt False}{\aFunction{x}{\aFunction{y}{y}}}
\desugar{
    {\tt if}~e_1~{\tt then}~e_2~{\tt else}~e_3~{\tt end}
}{
    (((e_1)(\aFunction{d}{e_2}))(\aFunction{d}{e_3}))(\aFunction{x}{x})
}
\desugar{{\tt while}~e_1~{\tt do}~e_2~{\tt end}}{
    {\tt function}~f()
}
%FIXED: I think e2;f should be e2;f().  Do you agree?
\desugarline{ \quad{\tt if}~e_1~{\tt then}~e_2;f()~{\tt else}~{\tt return}~{\tt nil}~{\tt end}\\
{\tt end};f()
}
\desugar{{\tt repeat}~e_1~{\tt until}~e_2}{
   {\tt while}~{\tt not}~e_2~{\tt do}~e_1~{\tt end}\\
}
%FIXME: I have some questions on the for loop -- let's discuss
\desugar{{\tt for}~x=e_1~{\tt ,}~e_2~{\tt ,}~e_3~{\tt do}~e_4~{\tt end}}{
    {\tt function}~f()
}
\desugarline{
    \quad {\tt local}~x = e_1;\\
    \quad {\tt while}~x<= e_2~{\tt do}~x = x + e_3;~e_4~{\tt end}\\
    {\tt end};f()\\
}
\end{array}\]
\end{figure}
