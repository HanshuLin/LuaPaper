\chapter{Featherweight Lua}
Although Lua has a minimal syntax compared to most other programming languages, it is still too complex for formal reasoning. Therefore, to make it simpler for reading and analyzing, we further simplify the syntax from the full version Lua. We call the new, simplified language {\it Featherweight Lua} (FWLua). In FWLua, we remove unnecessary syntax and reduce the language to its essential core.

In this chapter, we first introduce several primary parts in Lua --- expressions and statements, which can be simplified to FWLua in a straightforward way. Then, we are going to talk about functions and metatables, which are both key elements of Lua, in more detail in the chapter. Meanwhile, we will also give the full syntax and evaluation rules after all the elements in FWLua have been specifically introduced.

\section{Expressions}
Lua splits statements and expressions, though they overlap because of functions and function application. As a consequence, we merge expressions and statements, and treat both of them as expressions in FWLua. In this section, therefore, we discuss our expressions in FWLua and why we think these simplifications remain true to the spirit of Lua. 

\subsection{Lua's Expressions}

We keep constant values in the language.
These types include Number, String and Boolean representing values in the original syntax.
There is also a special expression {\tt nil}, similar to {\tt null} in Java and to both {\tt null} and {\tt undefined} in JavaScript.
For simplicity, we treat {\tt nil} as a constant in our evaluation rules.

In the original syntax,
there are several different ways for getting variables.
To make it concrete, we show a program using all three ways for getting variables:

\begin{verbatim}
tbl
tbl.x
tbl["y"]
\end{verbatim}

The first case references a variable name.
The second is a direct table select, using `.'.
This variable expression directly gets the value referenced by the key ``x'' in the specified table ``tbl''.
The last case is an example of a computed table access.
It is similar to the second case, except that Lua allows an expression to dynamically determine the key in the table.

The computed table access case can simulate the other approaches.
Referring to {\tt tbl.x} is identical to {\tt tbl["x"]}.
%FIXED: Mention which version of Lua added support for _ENV.
In the version of Lua 5.2, there is a new feature, which is using a reserved variable {\tt \_ENV} to represent the table of global environment.
Therefore, instead of referring to {\tt x}, we can use {\tt \_ENV.x}.

As a consequence, variables are now represented by a more common format: e\textsubscript{1}[e\textsubscript{2}], which we also call table select. In this format, we only look up the variable e\textsubscript{2} in the table e\textsubscript{1} and return the value (or nil if it doesn't exist). And we can use reserved word ``\_ENV'' to represent the global table. Specifically,

\begin{verbatim}
tbl
\end{verbatim}
is equal to
\begin{verbatim}
_ENV["tbl"]
\end{verbatim}

Table constructors
construct a new domain with some scoped mutable variables within.
When we invoke ``\{\}'', the compiler allocates a new scope
and creates a new anonymous table, similar to how {\tt new} works in JavaScript.

Lua supports both
unary operators and binary operators, though it is straightforward to treat unary operators as binary operators
with two expressions e\textsubscript{1} and e\textsubscript{2} where e\textsubscript{1} is {\tt nil}.
As a result, we eliminate unary operators from the syntax of FWLua.

The syntax of expressions in Featherweight Lua is presented and discussed in Section \ref{sec: FWLUAsyntax}

\subsection{Lua's Statements}
According to Roberto Ierusalimschy~\cite{PIL},
statements in Lua include assignment, control structures, function calls, and variable declarations.

%FIXED: Are these your words?  Let's discuss at the next meeting.
Our strategy for simplifying statements is to classify them into some syntactic groups, and then pick up the basic syntactic form for each of them.
We also remove blocks from our syntax since we can desugar sequences of statements easily.

We first discuss assignment statements.
Every assignment statement changes the store. 

Lua is also built to be a dynamic typed programming language, 
which means type declarations are not needed in assignment statements.
Based on this, we can merge all types of assignment statements into one single form. According to the original syntax, there might be some differences in local assignment statements.
For instance:

\begin{verbatim}
a = 3;   --A single assignment statement
b, c, d = 1, 2, 3;   --Multiple assignment statments
e, f = `hello', {};   --Handles different types
\end{verbatim}

There is also a form of assignment that allows using a list of expressions as either variables or values to complete multiple assignments in one statement. However, we choose the basic assignment form without lists or local scope.  The other forms can be produced with syntactic sugar.
In Chapter~\ref{chp:luaTranslation} we review the desugaring process in more depth.

Lua also features a large number of control structures.
However, these can be represented with functions,
following patterns established for the lambda calculus~\cite{TAPL}.
For instance, the following are some different kinds of control structures in Lua:
\begin{verbatim}
x = 0
while x ~= 10 do --A while loop
  x = x + 1
  print(x)
end

i = 0
repeat  --A repeat loop
  i = i + 1
  print(i)
until
i == 10

if (x == 0) then print("HelloWorld") --An if statement
else print("ByeWorld")
end
\end{verbatim}

Also, they can be represented using a set of lambda functions.
We will further discuss it in the Chapter \ref{chp:luaTranslation}.
Consequently, we eliminate these control statements from FWLua.

\section{Integration}
Finally, we treat every statement as an expression for simplicity in FWLua.
However, this means that FWLua is not a true subset of Lua.
Consider the following expressions:

\begin{verbatim}
3;   --A constant value
x;   --A variable
a = b = 10;   --Multiple statements
\end{verbatim}

Lua does not allow constants or variables as stand-alone statements,
and chained assignments are not supported in Lua.
While these are departures from Lua, we feel that they are minor discrepancies between the languages.

\section{Functions}
Functions are used to complete a set of computations and return a value.
To define a function in Lua, each of the following statements can be used:
\begin{verbatim}
function foo(x) x = x + 1 return x end    --normal function definition
foo = function (x) x = x + 1 return x end --using anonymous function
\end{verbatim}
We can invoke this function as follows:
\begin{verbatim}
foo(4);
\end{verbatim}
In addition, functions in the full version of Lua provide multiple expressions in the function, both in defining arguments and returning values. For example:
\begin{verbatim}
function add(x, y)
  local i = x + y;
  return i;
end
\end{verbatim}
and also:
\begin{verbatim}
function inc(x, y)
  x = x + 1;
  y = y + 1;
  return (x,y);
end
\end{verbatim}
We can see that a function can both take more than one argument and return multiple values.

However, we want to keep the syntax for FWLua as simple as we can. Therefore, we represent multiple expressions through desugaring, which we discuss later in this chapter.

Based on the examples above, we then choose to use the syntax of anonymous functions as our primary function syntax in FWLua. Since we have already come up with the syntax of assignment statements, we will combine it with anonymous function definition to define a function with a name. Besides, we restrict functions to one argument and one return value for decreasing unnecessary complexity in the syntax.

However, we have also raised the issue against one special statement called ``return statements''. As we mentioned above, return statements cannot really be called statements. Although Lua allows it appear individually like statements, return statements can only appear as the last statement in a block, and that is the reason why we need to treat them as a special cases in functions. In solving this issue, we decide to make the reserved word {\tt return} as the tail in the function and will use {\tt return nil} instead of no return statement in the block.

As the result, the syntax of function would be like the following:
\begin{verbatim}
function x return e end
\end{verbatim}
In the syntax, {\tt x} represents the argument a function takes (only one allowed), and {\tt e} represents the function body. Functions in FWLua still allow multiple expressions in the body of function through desugaring, outlined in Chapter~\ref{chp:luaTranslation}.

We will talk about some key factors related to functions in the following sections, using some examples to make them easier to comprehend.

%FIXED: Scoping and closures are different issues.  You need to divide this section up into two parts.
\subsection{Scoping}
Before talking about closures in Lua and FWLua, we will show some examples to help understand closures in functions.
\begin{verbatim}
x=42
function foo()
  print(x)
end
function bar()
  local x=100
  foo()
end
bar() --will print 42
\end{verbatim}
The above example shows that Lua uses static or lexical scoping.
Lexical scoping means that a stack containing variables in one's outer scope is defined once a function or a block is declared.
On the other hand,
%FIXED: Need to revise the definition of dynamic scoping.  Let's discuss.
with dynamic scoping, the compiler goes throughout the whole scope link, and thus find the variable that is being called.
The above example is testing Lua's scoping mechanism.
Once the function {\tt foo()} is invoked in function {\tt bar()}, Lua finds the value of variable {\tt x} from the stack of function {\tt foo()}, and then find its outer scope.
It means the scope in Lua is static and immutable. That is the reason why we also call this ``static scope''.
In contrast, if Lua used dynamic scoping, the above example would print 100.

%------MODEIFIED-------
\subsection{Closures}
Closures in a programming language enable functions to access variables in their outer scope. Closures is as important as scoping. It is commonly discussed together with scoping.

For a concrete understanding of closures, consider the following example:
%------MODEIFIED-------
\begin{verbatim}
function foo()
  local x = 0;
  return 
    function() 
       x = x + 1;
       return x; 
    end;
end
bar = foo();
print(bar()) --return 1
print(bar()) --return 2
print((foo())()) --return 1
\end{verbatim}

This shows closures in Lua.
In the example, once the variable {\tt bar} is declared, a stack storing scoping information is created.
%This stack is only about {\tt bar}, even if it is highly related to function {\tt foo}.
%FIXME: Revise this next sentence on bar and foo -- the key point is that foo() returns a closure.
In the condition, calling {\tt bar} and {\tt foo()} seems the same. However, they will point to two independent stack in inner environment, just as the above instance shows.

%In the case of lexical scoping and closures, we build our syntax of FWLua very static.
For setting variables, we need very exact directions of where FWLua should set values. This is a very interesting and important character in FWLua. We will discuss this issue more in Section~\ref{sec: FWLUAsemantic}, and then introduce our translation from Lua to FWLua in Chapter~\ref{chp:luaTranslation}.

\subsection{Recursion}\label{sec:recursions}

Recursion means that a function can call itself in its body.
Here, we give an example of a recursive function in Lua, {\tt factorial(n)}.

\begin{verbatim}
function factorial(n)
  if n == 0 then
    return 1;
  else
    return n * factorial(n-1);
  end
end

print(factorial(5)); --results 120
\end{verbatim}

In the example, we can see that only defining the function {\tt factorial(n)} is not simply enough. For implementing the factorial, it must be called with different argument in the body of itself, and this would be the classic recursive function.

Lua uses a call stack to store local variables in one function, and thus implement recursive functions~\cite{begLua}. Theoretically, when a function itself is called in its body, the interpreter will automatically treat it as just a variable and will further finish it after the function is done. In other words, multiple calls to the same function can be active at the same time without the crashes reporting ``undefined'' in Lua.

This quality in Lua is very helpful for us to give the further abstract syntax in FWLua about function, since functions in FWLua would be more like lambda calculus. Next, we will briefly introduce lambda calculus and a key factor for the recursion in lambda calculus called the fix combinator.

\subsection{Lambda calculus}
The restriction of allowing a function in FWLua to only take one argument comes from the lambda calculus. The method of lambda calculus is a formal algorithm representing computations based on functions. Function abstraction and function application are the two main parts in the syntax of lambda calculus. 

%FIXED: Still need to discuss this section
%---------MODIEFIED-----------
There are several advantages using lambda calculus: it can capture the essence of functional programming language; it has a minimal syntax while is able to build a very complex features. Functions in FWLua will use the system of lambda calculus. It is because lambda calculus is so powerful that enables us to remove many features, which can be represented using lambda calculus, from the full version of Lua. We will further show the detail in the section \ref{sec:TraFunc}.

In the very basic lambda calculus, the symbol `$\lambda$' means a function, the letter appears after `$\lambda$' means the argument this function takes.
%Actually, these two letters form the basic called function.
Besides, lambda calculus uses another expression as the body of function to form the function abstraction. What is more, in representing a function application, there is also an extra letter behind the function abstraction meaning ``the perimeter this function applies''. We can represent functions in FWLua using lambda calculus. For instance, the function call in FWLua:
%---------MODIEFIED-----------

\begin{verbatim}
(function(x) return x end)(a)
\end{verbatim}
can be represented as a function application ({\tt $\lambda$x.x a}) using lambda calculus. In the example, the expression {\tt $\lambda$x.x} means the function abstraction, taking argument {\tt x} and returning {\tt x} as the result. Furthermore, the letter {\tt a} shows the variable that this function applies.

%Therefore, it obviously proves the reason why we let
Although we only allow one argument in a function, functions with multiple arguments in Lua are common. Lambda calculus also gives the solution about this. Based on the syntax, the function taking 2 arguments ({\tt x} and {\tt y}) can be shown as {\tt $\lambda$x.$\lambda$y.y}. We will further talk about how we do the syntactic desugaring according to the basic lambda calculus.

\newcommand{\abFunction}[2]{{\tt function} ~{#1}~{\tt return}~{#2}~{\tt end}}
\newcommand{\semanticFullRaw}[4]{{#1},{#2} \Downarrow {#3},{#4}}
\newcommand{\semanticFull}[4]{{#1},{#2} \Downarrow {#3}, {#4}}


\section{Metatables and metamethods}
Metatables allow us to define behaviors as fallbacks when Lua cannot handle an operation. These operations even include symbols whose behavior we think of as being fixed, such as ``+'' or ``-''.
Also, metamethods in metatables can be automatically invoked once the relative conditions are satisfied. In other words, the behaviors of tables can be further controlled to follow our new rules in the program by using metatables. That is the reason why metatables are such an important characteristic of Lua.

However, metatables can only be set for tables. We cannot assign a metatable to some constant values, like numbers and strings. Those data types are too basic, and changing behaviors of some operator for them will mess up the world. At least, the authors do not want ``1 + 1'' results in 3. 

Although metatables are a core element in Lua, we represent their behavior as part of the desugaring process, rather than bringing it into the basic semantics of FWLua.
It is because we want to make FWLua as clean and simple as possible.
We encode metatables in the translation phase, which means metatables can be further translated as a special inner table when we desugar Lua into FWLua.

\subsection{Introduction about metatables}
Lua create a table value without any metatable by default.
However, we can set a metatable to tables.
There are two reserved functions related to metatables: {\tt setmetatable()} and {\tt getmetatable()}. These two functions are used for setting and getting a metatable of a table. 
Lua allows users to set any table as a metatable of other tables. Or, there is a need of C code to manipulate the metatables using other types. However, we will skip studying C in the paper and only discuss the former condition.

\subsection{What's in metatables}
Metamethods are reserved functions with a set of special names in a metatable. These functions can be automatically triggered, hence they are called ``hooks''. Actually, all the computations during programming can be thought as hidden functions. One of the purposes of metamethods is to let developers change the behavior of the programming language to a certain degree.

%FIXED: Do you have a citation for the 4 different kinds of metamethods?
Furthermore, there are 4 different kinds of metamethods in Lua based on the reference \cite{PIL}: arithmetic, relational, library-defined, and table-access. According to their names, each of them carries functions toward different fields. The arithmetic and relational metamethods are mostly responsible for binary operations, and the other two are often for tables and reserved functions. Generally, values in Lua can only take arithmetic and relational metamethods, since what they defined would not change the normal behavior of the programming language. On the other hand, table-access metamethods will possibly change the behavior of tables for several situations.

Figure \ref{fig:metatables} introduces the metamethods that play key roles in FWLua and then gives the type of each metamethod.


% -------------------------------[FW2.1][METATABLES]-------------------------------
\begin{figure}[P]
\caption{Metatable Event Types}
{\bf Table-Access Metamethods}
\label{fig:metatables}
\[
\begin{array}{rclcl}
  {\tt \_\_index}(get) & :: & {table} ~\rightarrow ~ {string} ~\rightarrow ~ {value} \\
  {\tt \_\_newindex}(set)   & :: &  {table} ~\rightarrow ~{string} ~\rightarrow ~{value} ~\rightarrow ~{\tt nil}\\
\end{array}
\]

{\bf Arithmetic Metamethods}
\[
\begin{array}{rclcl}
  {\tt \_\_add(+)}   & :: &  {value} ~\rightarrow ~{value}~\rightarrow ~{value}\\
  {\tt \_\_sub(-)}   & :: &  {value} ~\rightarrow ~{value}~\rightarrow ~{value}\\
  {\tt \_\_mul(*)}   & :: &  {value} ~\rightarrow ~{value}~\rightarrow ~{value}\\
  {\tt \_\_div(/)}   & :: &  {value} ~\rightarrow ~{value}~\rightarrow ~{value}\\
  {\tt \_\_mod(mod)}   & :: &  {value} ~\rightarrow ~{value}~\rightarrow ~{value}\\
  {\tt \_\_pow({\wedge})}   & :: &  {value} ~\rightarrow ~{value}~\rightarrow ~{value}\\
\end{array}
\]

{\bf Relational Metamethods}
\[
\begin{array}{rclcl}
  {\tt \_\_eq(==)} & :: & {value} ~\rightarrow ~ {value} ~\rightarrow ~ {value}\\
  {\tt \_\_ne(\sim=)} & :: & {value} ~\rightarrow ~ {value} ~\rightarrow ~ {value}\\
  {\tt \_\_lt(<)}   & :: &  {value} ~\rightarrow ~{value} ~\rightarrow ~{value}\\
  {\tt \_\_le(<=)}   & :: &  {value} ~\rightarrow ~{value} ~\rightarrow ~{value}\\
  {\tt \_\_gt(>)}   & :: &  {value} ~\rightarrow ~{value} ~\rightarrow ~{value}\\
  {\tt \_\_ge(>=)}   & :: &  {value} ~\rightarrow ~{value} ~\rightarrow ~{value}\\
\end{array}
\]
\end{figure}

\newcommand{\aFunction}[2]{\lambda{#1}.{#2}}
