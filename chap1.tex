\chapter{Introduction of Lua}
Lua is an extension programming language. It is designed to support general procedural programming with data description facilities. As a embeddable scripting language, it is relatively small and has a concise syntax. However, it is also powerful, fast, and lightweight that makes it being used in different domains.

What is more, as an extension language, Lua has no ``main" method or program. All it does is being used as an embedded function in its ``host". Therefore, host programs can invoke some functions to execute a portion of the Lua code. The scripts of Lua can be easily used when using C/C++. Also, Lua can use functions of C/C++ in its own code since it is implemented as a library, written in clean C \ref{cleanc}, the common subset of Standard C and C++.

The compiler of Lua can be executed in almost all platforms and is also very small (less that 200k in common). The distribution includes a sample host program called Lua, which uses the Lua library to offer a complete, standalone Lua interpreter, for interactive or batch use. Nowadays, Lua has been used in many domains because of its benefits and we will talk about it by some instance in the following section.

Talking about the project, our core goal is to develop semantics for a minimal version of Lua. This semantics allows us to formally reason about properties of the Lua language without becoming needlessly mired in unimportant aspects of the full Lua language. In other words, we are attempting to remove much stuff and keep those very core things in Lua, for providing an easier way in case of researching in the future.

\section{Where do we use Lua}
Lua is often embedded in many applications to fulfill some simple and small tasks because of its qualities. Usually, Lua is used to do much extending work in massive programs, which would be hard to change. Besides, there are also many programs about security start to use Lua so far. The truth we want to talk is that Lua can be used in many programs in different domains, which in need of many specific extending works to make them flexible. We will raise some examples in this section.

For instance, there are many user interface add-ons defined by Lua script in a very popular online game called The World of Warcraft\cite{wowLua}. These add-ons are commonly used for assisting players by adding some new simple functions in the game, such like timers, alerts, chat reminders, action automations and so on. In this massive game, each add-on that made by Lua is exquisite, and this obviously shows many qualities of Lua -- embedded, simple and lightweight.

Actually, Lua proves very useful in many areas and is becoming more fashion since such simple and agile language is in the trend of programming. In addition, for its flexibility, applications using ua could have a good ability of extending and customizing.

\section{Related Work beside Lua}
As we mentioned above, Lua is implemented as a library and written in clean C. Basically it will be easy for us to do further research about Lua because of this. However, we then find that Lua is more like the language called JavaScript. JavaScript is a very popular language with some benefits, which are same as Lua.

There once is a paper, talking about holding a new Language based on JavaScript called Featherweight JavaScript (FWJS). In the paper, dynamic information flow is focused. Also, it has mentioned several key elements about JavaScript, including the whole syntax and semantics about FWJS and how did they do that from original JavaScript. What we interested is the way they do desugaring using lambda calculus and a language called LambdaScript. There are many tricks and tips that are truly helpful for this project.