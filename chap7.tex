\chapter{Conclusion \& Future work}
All in all, this paper shows the syntax and semantics of the full version of Lua due to our analysis, and thus a simpler version of Lua that we call Featherweight Lua. FWLua removes a lot of elements and only retains the key components from Lua. We have focused on the exotic and important features, which are metatables and metamethods. During the research, we found that those factors are very powerful for Lua since they can change the behavior of the programming language. They can even model many different kinds of data structures combining the primary data structure in Lua --- tables.

In addition, we also give an approach to desugar Lua into FWLua. Due to this, users can gain a clearer understanding of the full version of Lua through this core language. We also give an interpreter implemented in Haskell as the implementation. This implementation proves to be very useful during our studying about the relationship between Lua and FWLua. With this implementation, we are able to test and verify our translations and semantics.